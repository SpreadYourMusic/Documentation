% Chapter Template

\chapter{Análisis y diseño del sistema} % Main chapter title

\label{Chapter4} % Change X to a consecutive number; for referencing this chapter elsewhere, use \ref{ChapterX}

\section{Análisis de requisitos}
Completar y detallar los requisitos preliminares incluidos en la propuesta técnica y económica. Recordad que los requisitos deben ser completos, concretos, medibles cuando tenga sentido y lo menos ambiguos posible. También es importante que estén identificados para facilitar su trazabilidad.

\section{Diseño del sistema}
\begin{itemize}
	\item Diagramas arquitecturales (de módulos, de componentes y conectores, de distribución), patrones de diseño y estilos arquitecturales que se aplicarán. Las interfaces (de módulos y de componentes) son especialmente importantes. También lo son los protocolos de comunicación entre componentes.
	\item Tecnologías elegidas (lenguajes de programación, componentes que se integrarán, API web externas con las que se conectará etc.).
	\item Otros aspectos técnicos de interés (p.ej. si hay base de datos si va a ser SQL o NoSQL, si hay una API Web va a ser RESTful o no, si algunas de las operaciones van a ser asíncronas o no, si va a ser una aplicación móvil o de escritorio será nativa o se van a usar tecnologías web, cómo se van a considerar los requisitos de seguridad o de prestaciones, cómo y dónde se harán las instalaciones y despliegues etc.)
\end{itemize}

Hay que justificar todas las decisiones de diseño. Esto exige contestar a dos preguntas sobre cada decisión: ¿qué alternativas se barajaron? y ¿por qué se eligió una y no las otras?
