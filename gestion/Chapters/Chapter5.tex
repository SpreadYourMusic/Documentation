% Chapter Template

\chapter{Memoria del proyecto} % Main chapter title

\label{Chapter5} % Change X to a consecutive number; for referencing this chapter elsewhere, use \ref{ChapterX}


ESTE CAPÍTULO NO SE RELLENA EN LA PRIMERA ENTREGA
En este capítulo se describirá cómo se ha llevado a cabo el proyecto, qué cambios se han hecho respecto a la versión inicial, imprevistos surgidos, etc.
\section{Inicio del proyecto}
Describir cómo transcurrió esta fase del proyecto, especialmente los resultados de llevar a cabo los procesos descritos en la sección Procesos de inicio del proyecto.
\section{Ejecución y control del proyecto}
Describir cómo transcurrió esta fase del proyecto, especialmente los resultados de llevar a cabo los procesos descritos en la sección Procesos de ejecución y control del proyecto y en la sección Procesos técnicos. No olvidar:
\begin{itemize}
	\item Cómo se ha realizado el reparto de trabajo entre miembros del equipo. Cómo ha transcurrido la comunicación interna.
	\item Cómo se ha medido el progreso del proyecto. Cómo se sabía el trabajo realizado, el trabajo pendiente y lo que estaba haciendo cada persona.
	\item Los ajustes realizados cuando se detectaron divergencias frente al calendario inicial (ajustes en el trabajo y/o ajustes en el calendario). Si se han identificado las causas de estas divergencias, explicarlas.
	\item Adecuación de las herramientas y tecnologías empleadas. Si ha habido que cambiar alguna decisión de diseño o de tecnología, y por qué.
	\item Funcionamiento de los procesos de control de versiones del código, construcción y despliegue. ¿Ha habido problemas con las integraciones? ¿Problemas con los despliegues? ¿Se han perdido cosas por errores humanos? ¿Cómo se han abordado estas tareas?
	\item Pruebas del software. ¿Se han podido cumplir las ideas que se tenían al respecto?
\end{itemize}

\section{Cierre del proyecto}
Al menos:
\begin{itemize}
	\item Comparar las estimaciones iniciales (tamaño, esfuerzos, costes) con los resultados finales, analizar los resultados y tratar de expresar algunas lecciones aprendidas.
	\item Lecciones aprendidas sobre herramientas y tecnologías.
	\item Recopilar los esfuerzos dedicados al proyecto por cada uno de los participantes: horas trabajadas y actividades realizadas por cada persona.
\end{itemize}