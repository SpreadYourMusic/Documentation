\documentclass[12pt]{article}%
\usepackage{amsfonts}
\usepackage{fancyhdr}
\usepackage{comment}
\usepackage[a4paper, top=2.5cm, bottom=2.5cm, left=2.2cm, right=2.2cm]%
{geometry}
\usepackage{times}
\usepackage{placeins}
\usepackage{amsmath}
\usepackage{changepage}
\usepackage{amssymb}
\usepackage{tikz}
\usepackage{float}
\usepackage{graphicx}%

\usepackage{listings}
\usepackage{color}

\definecolor{mygreen}{rgb}{0,0.6,0}
\definecolor{mygray}{rgb}{0.5,0.5,0.5}
\definecolor{mymauve}{rgb}{0.58,0,0.82}


\setcounter{MaxMatrixCols}{30}
\newtheorem{theorem}{Theorem}
\newtheorem{acknowledgement}[theorem]{Acknowledgement}
\newtheorem{algorithm}[theorem]{Algorithm}
\newtheorem{axiom}{Axiom}
\newtheorem{case}[theorem]{Case}
\newtheorem{claim}[theorem]{Claim}
\newtheorem{conclusion}[theorem]{Conclusion}
\newtheorem{condition}[theorem]{Condition}
\newtheorem{conjecture}[theorem]{Conjecture}
\newtheorem{corollary}[theorem]{Corollary}
\newtheorem{criterion}[theorem]{Criterion}
\newtheorem{definition}[theorem]{Definition}
\newtheorem{example}[theorem]{Example}
\newtheorem{exercise}[theorem]{Exercise}
\newtheorem{lemma}[theorem]{Lemma}
\newtheorem{notation}[theorem]{Notation}
\newtheorem{problem}[theorem]{Problem}
\newtheorem{proposition}[theorem]{Proposition}
\newtheorem{remark}[theorem]{Remark}
\newtheorem{solution}[theorem]{Solution}
\newtheorem{summary}[theorem]{Summary}
\newenvironment{proof}[1][Proof]{\textbf{#1.} }{\ \rule{0.5em}{0.5em}}

\newcommand{\Q}{\mathbb{Q}}
\newcommand{\R}{\mathbb{R}}
\newcommand{\C}{\mathbb{C}}
\newcommand{\Z}{\mathbb{Z}}

\usetikzlibrary{arrows.meta,
	quotes
}
\usetikzlibrary{automata,positioning}

\lstset{ 
	backgroundcolor=\color{white},   % choose the background color; you must add \usepackage{color} or \usepackage{xcolor}; should come as last argument
	basicstyle=\footnotesize,        % the size of the fonts that are used for the code
	breakatwhitespace=false,         % sets if automatic breaks should only happen at whitespace
	breaklines=true,                 % sets automatic line breaking
	captionpos=b,                    % sets the caption-position to bottom
	commentstyle=\color{mygreen},    % comment style
	deletekeywords={...},            % if you want to delete keywords from the given language
	escapeinside={\%*}{*)},          % if you want to add LaTeX within your code
	extendedchars=true,              % lets you use non-ASCII characters; for 8-bits encodings only, does not work with UTF-8
	frame=single,	                   % adds a frame around the code
	keepspaces=true,                 % keeps spaces in text, useful for keeping indentation of code (possibly needs columns=flexible)
	keywordstyle=\color{blue},       % keyword style
	language=C,		                 % the language of the code
	morekeywords={*,...},            % if you want to add more keywords to the set
	numbers=left,                    % where to put the line-numbers; possible values are (none, left, right)
	numbersep=5pt,                   % how far the line-numbers are from the code
	numberstyle=\tiny\color{mygray}, % the style that is used for the line-numbers
	rulecolor=\color{black},         % if not set, the frame-color may be changed on line-breaks within not-black text (e.g. comments (green here))
	showspaces=false,                % show spaces everywhere adding particular underscores; it overrides 'showstringspaces'
	showstringspaces=false,          % underline spaces within strings only
	showtabs=false,                  % show tabs within strings adding particular underscores
	stepnumber=2,                    % the step between two line-numbers. If it's 1, each line will be numbered
	stringstyle=\color{mymauve},     % string literal style
	tabsize=2,	                   % sets default tabsize to 2 spaces
	title=\lstname                   % show the filename of files included with \lstinputlisting; also try caption instead of title
}

\newcommand{\addCode}[2]{
	\lstinputlisting[caption=#2]{Code/#1.c}
}

\begin{document}

\title{Plan de gesti\'on, an\'alisis, dise\~no y memoria del proyecto}
\author{Spread Your Music}
\date{\today}
\maketitle

\tableofcontents

\newpage

%% Section 1

\section{Introducci\'on}

\subsection{Resumen}

Resumen del proyecto, prop\'osito, alcance, objetivos, entregables e hitos principales. Alrededor de una p\'agina es suficiente.

Incluir una breve descripci\'on de la estructura del resto del documento.

\subsection{Objetivos}

%% Section 2

\section{Organizaci\'on del proyecto} % Main chapter title
\subsection{Equipo}

\begin{itemize}
	\item Equipo del proyecto: integrantes del mismo, roles y responsabilidades. Qu\'e hace dentro del proyecto cada miembro del equipo. Aunque es normal que todo el mundo haga varias cosas, tambi\'en es importante que haya responsables definidos para las tareas importantes.
	
	\begin{itemize}
		\item Es importante determinar un director o directora de proyecto.
	\end{itemize}
\end{itemize}

%% Section 3

\section{Plan de gesti\'on del proyecto} % Main chapter title

\subsection{Procesos}

Aqu\'i se describe c\'omo se llevar\'an a cabo distintas tareas que hay que realizar en distintos momentos del proyecto.

\subsubsection{Procesos de inicio de proyecto}

\begin{itemize}
	\item C\'omo se van a identificar y asignar recursos (p.ej. conseguir servidores en cloud o tel\'efonos m\'oviles para pruebas, pero tambi\'en registrarse para acceder a API que se quieran integrar o a herramientas online que se quieran usar etc.).
	\item C\'omo se va a abordar la formaci\'on inicial de los miembros del equipo (revisar qu\'e tecnolog\'ias se van a usar, qu\'e componentes se van a integrar, con qu\'e API hay que conectar y qui\'enes tienen que formarse, o auto-formarse, en todas esas cosas y de qu\'e manera (hacer alg\'un curso online, planificar algo de tiempo para auto-formaci\'on con tutoriales y documentaci\'on etc.).
\end{itemize}

\subsubsection{Procesos de ejecuci\'on y control del proyecto}
\begin{itemize}
	\item C\'omo se llevar\'an a cabo las comunicaciones internas, el registro de las decisiones tomadas en reuniones, la redacci\'on de las actas etc.
	\item C\'omo se van a determinar las tareas a realizar y el reparto de las mismas a integrantes del equipo en el d\'ia a d\'ia.
	\item C\'omo se abordar\'an los temas de gesti\'on del equipo (moral, resoluci\'on de disputas...).
	\item Qu\'e se va a hacer respecto a medidas de progreso y monitorizaci\'on del estado del proyecto (qu\'e se mira/mide, cada cu\'anto tiempo, qu\'e se hace si se detectan problemas de rendimiento o avance insuficiente o desviaciones respecto al plan inicial...).
	\item C\'omo se har\'a la entrega de resultados.
\end{itemize}

Com. internas: Mensajer\'ia instant\'anea y Issues de GitHub
Registro: Se tomar\'an actas y se subir\'an a un repo Git (cada acta incluye al menos fecha, hora, duraci\'on, miembros presentes, temas tratados.).
Las actas las redactar\'a un miembro del equipo y se tratar\'an como las actas de las reuniones del equipo.

Las tareas las determinar\'a el responsable del subproyecto y las asignar\'a a cada miembro

\subsubsection{Procesos t\'ecnicos}
\begin{itemize}
	\item Describir los m\'etodos, herramientas y t\'ecnicas necesarios tanto para construir el software (p.ej. herramientas de desarrollo), desplegarlo, probarlo (todos los necesarios para dar soporte a los planes descritos en la secci\'on 3.2).\\
\end{itemize}

\subsection{Planes}
\subsubsection{Plan de gesti\'on de configuraciones}
\begin{itemize}
	\item Convenciones de nombres (documentos) y est\'andares de c\'odigo.
	\item Responsable o responsables de las distintas actividades (puesta en marcha, apoyo al equipo, revisi\'on de commits, copias de seguridad, control de las versiones entregadas a cliente...).
	\item Recursos: repositorios de control de versiones (cu\'ales, cu\'antos, permisos de acceso a los mismos) y sistema de gesti\'on de incidencias.
	\item Procedimiento para realizar cambios al c\'odigo fuente y los documentos t\'ecnicos: workflow de control de versiones utilizado, cu\'ando/c\'omo se permiten realizar commits al repositorio compartido, si tienen que ser aceptados por alguien previamente o no, qu\'e hay que anotar en el sistema de gesti\'on de incidencias, qui\'en decide el estado de las incidencias, en qu\'e estados puede estar una incidencia etc.
\end{itemize}

\subsubsection{Plan de construcci\'on y despliegue del software}

\begin{itemize}
	\item C\'omo se construye e integra el software: si hay scripts de construcci\'on automatizada o no (en ese caso qu\'e se usa, y c\'omo se garantiza que todos los participantes compilan igual y con las mismas dependencias), qu\'e se incluye en la construcci\'on (descarga y actualizaci\'on de dependencias, compilaci\'on, ejecuci\'on de tests autom\'aticos...) y cada cu\'anto se construye (compila, integra, prueba) el sistema completo, c\'omo se configuran los computadores de los desarrolladores.
	\item C\'omo se despliega el software m\'as all\'a de las m\'aquinas de desarrollo: contenedores, m\'aquinas virtuales, servidor en cloud etc. y c\'omo se configuran esos entornos (rutas, usuarios y contraseñas, puertos y otros elementos).
\end{itemize}

\subsubsection{Plan de aseguramiento de la calidad}

\begin{itemize}
	\item Est\'andares de c\'odigo y otros (se pueden definir gu\'ias para la documentaci\'on de diseño y otros documentos del proyecto).
	\item Actividades de control de calidad del c\'odigo que se realizar\'an: revisiones de c\'odigo por pares, revisiones de requisitos o diagramas UML por pares, tipos de tests autom\'aticos o manuales que se llevar\'an a cabo.
\end{itemize}

\subsubsection{Calendario del proyecto y divisi\'on del trabajo}
\begin{itemize}
	\item Diagrama de Gantt que recoja las tareas a realizar. Tened en cuenta que trabaj\'ais con dos iteraciones y por tanto que hay una entrega intermedia y una final, y reflejarlo en este diagrama. Tened en cuenta que es normal que lo teng\'ais que actualizar conforme avance el proyecto (cu\'ando y c\'omo establezc\'ais en la secci\'on 3.1.2).
	\begin{itemize}
		\item Debe quedar claro qu\'e requisitos van a estar completados en la primera iteraci\'on y cu\'ales en la segunda. Es posible que para la primera iteraci\'on no se planifique completar ning\'un requisito, pero en ese caso tiene que planificarse qu\'e se har\'a y que faltar\'a por hacer para cada requisito.
	\end{itemize}
	\item Divisi\'on del trabajo en partes (los m\'odulos del software a desarrollar, pero tambi\'en  la documentaci\'on, el diseño gr\'afico, instalaciones o despliegues, pruebas manuales etc.) y reparto de los mismos entre el equipo de desarrollo, al menos a alto nivel (el reparto de labores concretas en el d\'ia a d\'ia no se detalla aqu\'i, pero hay que explicar bajo qu\'e criterios y qui\'en/c\'omo se hace en la secci\'on 3.1.2). Debe haber una correspondencia con las tareas que aparecen en el diagrama de Gantt (que no necesariamente tiene que ser una relaci\'on 1 a 1).
	\begin{itemize}
		\item Verificar que esta divisi\'on del trabajo cubre todos los requisitos.
	\end{itemize}
\end{itemize}


%% Section 4

\section{An\'alisis y diseño del sistema} % Main chapter title

\subsection{An\'alisis de requisitos}
Completar y detallar los requisitos preliminares incluidos en la propuesta t\'ecnica y econ\'omica. Recordad que los requisitos deben ser completos, concretos, medibles cuando tenga sentido y lo menos ambiguos posible. Tambi\'en es importante que est\'en identificados para facilitar su trazabilidad.

\subsection{Diseño del sistema}
\begin{itemize}
	\item Diagramas arquitecturales (de m\'odulos, de componentes y conectores, de distribuci\'on), patrones de diseño y estilos arquitecturales que se aplicar\'an. Las interfaces (de m\'odulos y de componentes) son especialmente importantes. Tambi\'en lo son los protocolos de comunicaci\'on entre componentes.
	\item Tecnolog\'ias elegidas (lenguajes de programaci\'on, componentes que se integrar\'an, API web externas con las que se conectar\'a etc.).
	\item Otros aspectos t\'ecnicos de inter\'es (p.ej. si hay base de datos si va a ser SQL o NoSQL, si hay una API Web va a ser RESTful o no, si algunas de las operaciones van a ser as\'incronas o no, si va a ser una aplicaci\'on m\'ovil o de escritorio ser\'a nativa o se van a usar tecnolog\'ias web, c\'omo se van a considerar los requisitos de seguridad o de prestaciones, c\'omo y d\'onde se har\'an las instalaciones y despliegues etc.)
\end{itemize}

Hay que justificar todas las decisiones de diseño. Esto exige contestar a dos preguntas sobre cada decisi\'on: ¿qu\'e alternativas se barajaron? y ¿por qu\'e se eligi\'o una y no las otras?


%% Section 5

\section{Memoria del proyecto} % Main chapter title

\label{Chapter5} % Change X to a consecutive number; for referencing this chapter elsewhere, use \ref{ChapterX}


ESTE CAP\'iTULO NO SE RELLENA EN LA PRIMERA ENTREGA
En este cap\'itulo se describir\'a c\'omo se ha llevado a cabo el proyecto, qu\'e cambios se han hecho respecto a la versi\'on inicial, imprevistos surgidos, etc.
\subsection{Inicio del proyecto}
Describir c\'omo transcurri\'o esta fase del proyecto, especialmente los resultados de llevar a cabo los procesos descritos en la secci\'on Procesos de inicio del proyecto.
\subsection{Ejecuci\'on y control del proyecto}
Describir c\'omo transcurri\'o esta fase del proyecto, especialmente los resultados de llevar a cabo los procesos descritos en la secci\'on Procesos de ejecuci\'on y control del proyecto y en la secci\'on Procesos t\'ecnicos. No olvidar:
\begin{itemize}
	\item C\'omo se ha realizado el reparto de trabajo entre miembros del equipo. C\'omo ha transcurrido la comunicaci\'on interna.
	\item C\'omo se ha medido el progreso del proyecto. C\'omo se sab\'ia el trabajo realizado, el trabajo pendiente y lo que estaba haciendo cada persona.
	\item Los ajustes realizados cuando se detectaron divergencias frente al calendario inicial (ajustes en el trabajo y/o ajustes en el calendario). Si se han identificado las causas de estas divergencias, explicarlas.
	\item Adecuaci\'on de las herramientas y tecnolog\'ias empleadas. Si ha habido que cambiar alguna decisi\'on de diseño o de tecnolog\'ia, y por qu\'e.
	\item Funcionamiento de los procesos de control de versiones del c\'odigo, construcci\'on y despliegue. ¿Ha habido problemas con las integraciones? ¿Problemas con los despliegues? ¿Se han perdido cosas por errores humanos? ¿C\'omo se han abordado estas tareas?
	\item Pruebas del software. ¿Se han podido cumplir las ideas que se ten\'ian al respecto?
\end{itemize}

\subsection{Cierre del proyecto}
Al menos:
\begin{itemize}
	\item Comparar las estimaciones iniciales (tamaño, esfuerzos, costes) con los resultados finales, analizar los resultados y tratar de expresar algunas lecciones aprendidas.
	\item Lecciones aprendidas sobre herramientas y tecnolog\'ias.
	\item Recopilar los esfuerzos dedicados al proyecto por cada uno de los participantes: horas trabajadas y actividades realizadas por cada persona.
\end{itemize}

%% Section 6

\section{Conclusiones} % Main chapter title


ESTE CAP\'iTULO SOLO SE RELLENA EN LA ENTREGA FINAL
Adem\'as de conclusiones personales (razonadas) sobre el transcurso del proyecto realizado, es importante plantear ideas para mejorar los procesos llevados a cabo: si hubiera que iniciar un nuevo proyecto inmediatamente usando una metodolog\'ia de gesti\'on basada en procesos, ¿qu\'e cambios har\'iais respecto a los procesos que hab\'eis seguido durante este proyecto?  ¿Qu\'e cosas est\'a claro que har\'iais de otra forma? ¿Qu\'e cosas seguir\'iais haciendo m\'as o menos igual?


\end{document}