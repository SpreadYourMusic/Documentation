% Chapter Template

\chapter{Objetivos del sistema} % Main chapter title

\label{Chapter2} % Change X to a consecutive number; for referencing this chapter elsewhere, use \ref{ChapterX}

%----------------------------------------------------------------------------------------
%	SECTION 1
%----------------------------------------------------------------------------------------

\section{Objetivos}

Descripción breve de las necesidades que el sistema debe abordar y la funcionalidad que se desarrollará para para cubrir estas necesidades.

\subsection{An\'alisis de requisitos preliminar}

Desgranar el punto anterior en forma de requisitos. No tiene que ser el análisis de requisitos totalmente completo y detallado, que se lleva a cabo una vez se confirma que el proyecto se va a llevar a cabo, pero es un compromiso con el cliente y debe incluir todo lo importante y estar adecuadamente redactado (con precisión, sin ambigüedades obvias, etc.). Además de texto, se pueden incluir algunos bocetos de la GUI (suelen ser importantes para determinar cómo va a ser una aplicación).