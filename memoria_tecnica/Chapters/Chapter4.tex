% Chapter Template

\chapter{Plan de trabajo} % Main chapter title

\label{Chapter4} % Change X to a consecutive number; for referencing this chapter elsewhere, use \ref{ChapterX}

%----------------------------------------------------------------------------------------
%	SECTION 1
%----------------------------------------------------------------------------------------

\section{Plan de trabajo}

Calendario preliminar del proyecto. No es el plan de trabajo detallado que tiene que hacer internamente el proveedor, sino un plan con las fechas e hitos relevantes para los clientes:

\begin{itemize}
	\item Qué se va a entregar (código, ejecutables, instaladores, manuales, software instalado en un servidor del que se transfieren las credenciales de acceso etc.).
	\item Cuándo se va a entregar cada cosa.
	\item Otros hitos de interés: reuniones previstas con el cliente, demostración de resultados intermedios, etc.
\end{itemize}
