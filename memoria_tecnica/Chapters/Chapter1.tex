% Chapter Template

\chapter{Resumen Ejecutivo} % Main chapter title

\label{Chapter1} % Change X to a consecutive number; for referencing this chapter elsewhere, use \ref{ChapterX}

%----------------------------------------------------------------------------------------
%	SECTION 1
%----------------------------------------------------------------------------------------

\section{Resumen}

Este es un breve análisis de los aspectos más importantes del proyecto, va antes de la presentación y es lo primero o a veces lo único que lee el receptor del documento, por lo tanto en pocas palabras se deben describir los elementos más relevantes de la propuesta. Con este resumen, lo que se busca es captar la atención del lector y motivarlo a leer el resto del documento. Son muchas las propuestas que son desechadas porque no se consigue superar este primer escalón. Por lo tanto debe estar bien redactado y presentado.\\

Algunas consideraciones:
\begin{itemize}
	\item El resumen debe caber en una página, salvo que incluya algún gráfico o diagrama.
	\item Se debe incluir un resumen del sistema a desarrollar. Se puede partir de la propia descripción que nos haya dado el cliente.
	\item Se deben incluir el precio, los plazos y los entregables.
\end{itemize}
