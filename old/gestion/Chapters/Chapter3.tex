% Chapter Template

\chapter{Plan de gestión del proyecto} % Main chapter title

\label{Chapter3} % Change X to a consecutive number; for referencing this chapter elsewhere, use \ref{ChapterX}

%----------------------------------------------------------------------------------------
%	SECTION 1
%----------------------------------------------------------------------------------------

\section{Procesos}

Aquí se describe cómo se llevarán a cabo distintas tareas que hay que realizar en distintos momentos del proyecto.

\subsection{Procesos de inicio de proyecto}

\begin{itemize}
	\item Cómo se van a identificar y asignar recursos (p.ej. conseguir servidores en cloud o teléfonos móviles para pruebas, pero también registrarse para acceder a API que se quieran integrar o a herramientas online que se quieran usar etc.).
	\item Cómo se va a abordar la formación inicial de los miembros del equipo (revisar qué tecnologías se van a usar, qué componentes se van a integrar, con qué API hay que conectar y quiénes tienen que formarse, o auto-formarse, en todas esas cosas y de qué manera (hacer algún curso online, planificar algo de tiempo para auto-formación con tutoriales y documentación etc.).
\end{itemize}

\subsection{Procesos de ejecución y control del proyecto}
\begin{itemize}
	\item Cómo se llevarán a cabo las comunicaciones internas, el registro de las decisiones tomadas en reuniones, la redacción de las actas etc.
	\item Cómo se van a determinar las tareas a realizar y el reparto de las mismas a integrantes del equipo en el día a día.
	\item Cómo se abordarán los temas de gestión del equipo (moral, resolución de disputas...).
	\item Qué se va a hacer respecto a medidas de progreso y monitorización del estado del proyecto (qué se mira/mide, cada cuánto tiempo, qué se hace si se detectan problemas de rendimiento o avance insuficiente o desviaciones respecto al plan inicial...).
	\item Cómo se hará la entrega de resultados.
\end{itemize}

\subsection{Procesos técnicos}
\begin{itemize}
	\item Describir los métodos, herramientas y técnicas necesarios tanto para construir el software (p.ej. herramientas de desarrollo), desplegarlo, probarlo (todos los necesarios para dar soporte a los planes descritos en la sección 3.2).\\
\end{itemize}

\section{Planes}
\subsection{Plan de gestión de configuraciones}
\begin{itemize}
	\item Convenciones de nombres (documentos) y estándares de código.
	\item Responsable o responsables de las distintas actividades (puesta en marcha, apoyo al equipo, revisión de commits, copias de seguridad, control de las versiones entregadas a cliente...).
	\item Recursos: repositorios de control de versiones (cuáles, cuántos, permisos de acceso a los mismos) y sistema de gestión de incidencias.
	\item Procedimiento para realizar cambios al código fuente y los documentos técnicos: workflow de control de versiones utilizado, cuándo/cómo se permiten realizar commits al repositorio compartido, si tienen que ser aceptados por alguien previamente o no, qué hay que anotar en el sistema de gestión de incidencias, quién decide el estado de las incidencias, en qué estados puede estar una incidencia etc.
\end{itemize}

\subsection{Plan de construcción y despliegue del software}

\begin{itemize}
	\item Cómo se construye e integra el software: si hay scripts de construcción automatizada o no (en ese caso qué se usa, y cómo se garantiza que todos los participantes compilan igual y con las mismas dependencias), qué se incluye en la construcción (descarga y actualización de dependencias, compilación, ejecución de tests automáticos...) y cada cuánto se construye (compila, integra, prueba) el sistema completo, cómo se configuran los computadores de los desarrolladores.
	\item Cómo se despliega el software más allá de las máquinas de desarrollo: contenedores, máquinas virtuales, servidor en cloud etc. y cómo se configuran esos entornos (rutas, usuarios y contraseñas, puertos y otros elementos).
\end{itemize}

\subsection{Plan de aseguramiento de la calidad}

\begin{itemize}
	\item Estándares de código y otros (se pueden definir guías para la documentación de diseño y otros documentos del proyecto).
	\item Actividades de control de calidad del código que se realizarán: revisiones de código por pares, revisiones de requisitos o diagramas UML por pares, tipos de tests automáticos o manuales que se llevarán a cabo.
\end{itemize}

\subsection{Calendario del proyecto y división del trabajo}
\begin{itemize}
	\item Diagrama de Gantt que recoja las tareas a realizar. Tened en cuenta que trabajáis con dos iteraciones y por tanto que hay una entrega intermedia y una final, y reflejarlo en este diagrama. Tened en cuenta que es normal que lo tengáis que actualizar conforme avance el proyecto (cuándo y cómo establezcáis en la sección 3.1.2).
	\begin{itemize}
		\item Debe quedar claro qué requisitos van a estar completados en la primera iteración y cuáles en la segunda. Es posible que para la primera iteración no se planifique completar ningún requisito, pero en ese caso tiene que planificarse qué se hará y que faltará por hacer para cada requisito.
	\end{itemize}
	\item División del trabajo en partes (los módulos del software a desarrollar, pero también  la documentación, el diseño gráfico, instalaciones o despliegues, pruebas manuales etc.) y reparto de los mismos entre el equipo de desarrollo, al menos a alto nivel (el reparto de labores concretas en el día a día no se detalla aquí, pero hay que explicar bajo qué criterios y quién/cómo se hace en la sección 3.1.2). Debe haber una correspondencia con las tareas que aparecen en el diagrama de Gantt (que no necesariamente tiene que ser una relación 1 a 1).
	\begin{itemize}
		\item Verificar que esta división del trabajo cubre todos los requisitos.
	\end{itemize}
\end{itemize}
