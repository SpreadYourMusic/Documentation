% Chapter Template

\chapter{Resumen Ejecutivo} % Main chapter title

\label{Chapter1} % Change X to a consecutive number; for referencing this chapter elsewhere, use \ref{ChapterX}

%----------------------------------------------------------------------------------------
%	SECTION 1
%----------------------------------------------------------------------------------------

\section{Resumen}

La aplicación a desarrollar consistirá en un repositorio de música. 

Algunas consideraciones:
\begin{itemize}
	\item El sistema a desarrollar es un reproductor de música en streaming inspirado en Soundcloud y Spotify. Para el se desarrollará una interfaz Web, una aplicación nativa de Android y el backend que sustenta a estas dos.
	\item El sistema permitirá a los usuario subir canciones, crear listas de reproducción, escuchar canciones utilizando un reproductor propio y permitiendo en la versión Android descargar las canciones para poder escuchar la música offline. También incluirá características sociales, permitiendo a los usuarios seguir a otros usuarios para ver las novedades que publican o suscribirse a listas de reproducción creadas por otros usuarios para enterarse de cambios en esta. El sistema también poseerá integración con redes sociales así como la posibilidad de autentificaci\'on mediante cuenta de Google.
	\item El propio sistema generará recomendaciones personalizadas para el usuario tanto basadas en su historial de reproducción como basadas en geolocalización y facilitará al usuario encontrar canciones, pudiendo buscar canciones por categorías, autor, nombre así como también mostrando canciones populares dentro de la aplicación.
	
	\item Se realizar\'an entregas del c\'odigo desarrollado hasta el momento los d\'ias 9 abril y 1 junio. En la segunda entrega se entregar\'a todo el c\'odigo necesario para cumplir con todos los requisitos especificados en el Cap\'itulo 2.
	\item El precio del sistema ser\'a de 13.161,61 euros.
\end{itemize}
