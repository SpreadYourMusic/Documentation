% Chapter Template

\chapter{Objetivos del sistema} % Main chapter title

\label{Chapter2} % Change X to a consecutive number; for referencing this chapter elsewhere, use \ref{ChapterX}

%----------------------------------------------------------------------------------------
%	SECTION 1
%----------------------------------------------------------------------------------------

\section{Objetivos}

Para la creación de contenido se permite el registro. Se contará con tanto con 3 perfiles de usuario: sin registro, con registro y administrador.
La aplicación cuenta a su vez con una version de Web y una version movil para Android con reproducción sincronizada entre ambos.
Los usuarios registrados tienen la posibilidad de administrar listas de reproducción públicas o privadas y a su vez el sistema genera listas de recomendaciones generales y para cada ususario.
A continuación se detalla el total de los requisitos.

\subsection{An\'alisis de requisitos preliminar}

\begin{itemize}
\item Requisitos funcionales
\begin{enumerate}
	\item Existen 3 perfiles de usuario: registrados, no registrados y administrador.
	\item La aplicación permite a los usuarios registrados subir canciones.
	\item La aplicación tiene un repositorio de canciones compuesto de las canciones de los usuarios.
	\item La aplicación permite reproducir canciones almacenadas en el repositorio.
	\item La aplicación permite gestionar listas de reproducción (se detalla posteriormente)
	\item - El sistema genera listas de reproducción segun las siguientes categorias: Género, Éxitos, Situación/Mood, País.
	\item - El sistema genera listas de reproduccion propias para cada usuario con recomendaciones según sus intereses.
	\item - El sistema genera listas de reproduccion con recomendaciones según geolocalización.
	\item - El sistema permite gestionar una lista privada de favoritos.

\end{enumerate}

\item Requisitos no funcionales
\begin{enumerate}
	\item La aplicación tendrá una versión para Android y otra para Web.
	\item El sistema soporta los ficheros MP3, WAV, OGG.
	\item Existe un servidor para el almacenamiento de canciones.
	\item Los datos que se recogen en el formulario de registro son los siguientes: Nombre, Nick, Correo, Contraseña, Fecha de nacimiento, Biografía, Redes Sociales, Foto de perfil(opcional), País
	\item Las canciones contienen los siguientes metadatos: Título, Idioma (opcional), número de reproducciones, número de 'Me gusta'.
	\item La aplicación movil soportará Android 5.0 y la Web la última versión de Firefox y Chrome.
\end{enumerate}

\end{itemize}

Android 

\begin{itemize}
\item Requisitos funcionales
\begin{enumerate}
\item El sistema permite un solo tipo de usuario, y es el usuario registrado (usuario que esté registrado en el sistema).
\item El sistema permite que los usuarios de identifiquen tanto con usuario y contraseña así como con una cuenta de Google.
\item El sistema permite que los usuarios suban canciones a la aplicación
\item El sistema permite que los usuarios se registren en la aplicación.
\item El sistema permite a los usuarios tener una lista de canciones favoritas, la cual será privada.
\item El sistema permite añadir a favorito una canción que se está escuchando.
\item El sistema permite desde cualquier pantalla escuchar canciones y ver los controles básicos de canciones (play, pause y pasar canción)
\item El sistema permite crear listas de reproducción formadas por canciones de las que puedes ser autor o no, las cuales serán publicas.
\item El sistema permite modificar los datos de usuario una vez creado.
\item El sistema posee una pantalla específica en la que se puede ver las canciones que posee una lista de reproducción así como seguirla.
\item El sistema permite buscar las canciones más populares de un género.
\item El sistema permite ver las canciones más populares en el país desde el que se conecta el dispositivo.
\item El sistema permite buscar canciones, artistas y listas de reproducción por su nombre mediante un buscador.
\item El sistema permite desde una pantalla específica acceder a más controles sobre las canciones.
\item - Desde la pantalla específica el sistema permite avanzar o retroceder en la canción a un momento exacto de esta.
\item - Desde la pantalla específica el sistema permite ver la letra de la canción dinámica (va al ritmo de la canción) en el caso que la posea.
\item - Desde la pantalla específica el sistema permite ver la onda de sonido de la canción.
\item - Desde la pantalla específica el sistema permite cambiar el orden de muestra de las canciones (aleatorio o lineal).
\item Se permite almacenar la música en el dispositivo para poder acceder a esta sin conexión a internet.
\end{enumerate}

\item Requisitos no funcionales 

\begin{enumerate}
\item El sistema posee una pantalla principal en la que se mostrarán recomendaciones (canciones, listas de reproducción y usuarios), novedades sobre los usuarios y listas de reproducción seguidos y canciones más populares. 
\item El sistema posee una pantalla de usuario (otros usuarios) en la que se puede ver su información, canciones, listas de reproducción creadas, su número de seguidores, la opción de seguirlo así como un enlace a sus redes sociales.
\item El sistema posee una pantalla de usuario (usuario propio) en la que se puede ver las canciones del propio usuario y sus listas de reproducción creadas, así como añadir más canciones, listas de reproducción o eliminar alguna de las dos.
\end{enumerate}
\end{itemize}

Web

\begin{itemize}

\item Requisitos no funcionales
\begin{enumerate}
\item El sistema tiene una pantalla principal donde se mostraran novedades de artiastas y canciones y canciones mas populares.
\item El sistema tiene una pantalla de ususario pública para todo el mundo donde aparece su información, canciones, listas de reproducción públicas, número de seguidores y enlaces a sus redes sociales
\item El sistema cuenta en todas las pantallas con un reproductor de audio que incluye información sobre la canción que se esta reproduciendo y las opciones de avanzar, retroceder a un momento de la canción; avanzar o retroceder a la siguiente o anterior canción si se esta reproduciendo una lista; parar la reproducción y cambiar el orden de reproducción a aleatorio.
\item El sistema tiene en todas las pantallas un buscador para realizar una búsqueda de canciones, listas y usuarios.
\item El sistema tiene una pantalla para cada canción donde se puede ver la  información de la canción, la onda de sonido y si tuviese las letras de \item forma dinámica si se estuviese reproduciendo.
\item El sistema tiene una pantalla para identificarse/ registrarse.
\end{enumerate}

\item Requisitos funcionales 
\begin{enumerate}
\item El sistema permite registrarse mediante usuario o a través de Google.
\item Verificación de cuenta. Esto supone un indicador gráfico en el perfil de usuario que anteriormente a verificado su identidad. La finalidad es garantizar la verdadera identidad del usuario.
\item La reproducción actual de un usuario se sincroniza en todos los dispositivos.

\item Existen 3 perfiles de usuarios: registrados, no registrados y administrador.
    \begin{enumerate}
	\item Registrados
	    \begin{enumerate}
	        \item Los usuarios no registrados pueden acceder a la página principal, páginas de los diferentes usuarios, páginas de canciones ademas de poder realizar búsquedas.
	        \item Los usuarios no registrados pueden reproducir canciones de usuarios o de listas de reproducción públicas.
	
	    \end{enumerate}
        \item Registrados
	    \begin{enumerate}
		\item Los usuarios registrados pueden realizar todas las funcionalidades de un usuario no registrado.
	        \item El sistema permite a los usuarios registrados identificarse.
	        \item El sistema permite a los usuarios registrados modificar su información personal.
	        \item El sistema permite a los usuarios registrados subir canciones.
	        \item El sistema permite administrar una lista de reproducción privada de favoritos.
	        \item El sistema permite añadir cualquier canción a favoritos.
	        \item El sistema permite crear listas de reproducción públicas
	        \item El sistema permite añadir cualquier canción a una lista de reproducción previamente creada por el usuario.
	        \item El sistema permite eliminar una canción de sus listas de reproducción o de la lista favoritos.	
	        \item El sistema permite seguir/ dejar de seguir a un usuario.
 	        \item El sistema permite seguir/ dejar de seguir a una lista de reproducción.
	        \item El sistema permite modificar sus enlaces a redes sociales.
		\item (RNF) La pantalla principal de un usuario registrado contiene recomendaciones de artistas, listas o usuarios en base a sus gusto y novedades de sus artistas/listas seguidos.
            \end{enumerate}
	\item Administrador
            \begin{enumerate}
		\item El usuario administrador se identifica en la aplicación con un usuarios y contraseña predefinidos.
	        \item El sistema permite al usuario administrador puede verificar la identidad de un usuario.
	        \item El sistema permite al usuario administrador modificar la información de un usuario, canción o playlist.
	        \item (RNF) La pantalla de administrador incluye un buscador para la búsqueda de un usuario, lista o canción y una lista con el resultado de la búsqueda.
	        \item (RNF) Cada canción, lista o usuario tiene una pantalla de administrador desde donde el sistema permite al administrador la modificación.
	
	    \end{enumerate}
    \end{enumerate}
\end{enumerate}

\end{itemize}


\subsection{Prototipo pantallas Android}
% ______________________
% Aqui comienza la tabla
\begin{tabular}{ p{6cm} p{6cm}}
\hline
\\
\includegraphics[width=6cm]{Figures/android/Login.png}
&
\includegraphics[width=6cm]{Figures/android/home.png} \\
\hline
\\
Pantalla de autentificación, en la que se permite tanto el acceso desde un usuario y contraseña de la aplicación así como desde una cuenta de Google. Esta es la primera pantalla que se abre en el caso de que el usuario no esté conectado
&
Pantalla principal de la aplicación en la cual se mostrarían diversos elementos de interés para el usuario, como pueden ser recomendaciones, novedades así como nuevas publicaciones que han producido artistas a los que se está suscrito.
Este esquema de pantalla también lo comparte una pantalla llamada explorar, la pantalla de novedades y la pantalla de recomendaciones.
En la pantalla explorar se mostrarían diversas listas de reproducción generadas por el sistema ( listas basadas en la ubicación del usuario, basadas en lo que escucha el usuario, basadas en lo que escucha el conjunto total de usuarios) así como los diferentes géneros musicales.  \\
\hline
\end{tabular}

\begin{tabular}{ p{6cm} p{6cm}}
\hline
\\
\includegraphics[width=6cm]{Figures/android/Home-NavigationDrawer.png}
&
\includegraphics[width=6cm]{Figures/android/lista-reproductor.png} \\
\hline
\\
La forma de ir de una pantalla a otra en la aplicación sería mediante un panel lateral.
&
Desde cualquier pantalla se puede reproducir música, no es necesario que se esté en la pantalla del reproductor \\
\hline
\end{tabular}

\begin{tabular}{ p{6cm} p{6cm}}
\hline
\\
\includegraphics[width=6cm]{Figures/android/lista-no-personal.png}
&
\includegraphics[width=6cm]{Figures/android/lista-personal.png} \\
\hline
\\
Lista de canciones, artistas o playlists. Este esquema de pantalla sería el que se produce al realizar una búsqueda o al acceder a las distintas categorías de mi biblioteca (playlist a las que sigues, canciones que te han gustado, artistas a los que sigues y canciones descargadas).
&
Este esquema de pantalla sería el que se produce al acceder a “Mis canciones” o “Mis playlists”.
Difiere del anterior en que permite añadir más elementos desde la pantalla. \\
\hline
\end{tabular}

\begin{tabular}{ p{6cm} p{6cm}}
\hline
\\
\includegraphics[width=6cm]{Figures/android/lista-reproductor.png}
&
\includegraphics[width=6cm]{Figures/android/Reproductor.png} \\
\hline
\\
Desde cualquier pantalla se puede reproducir música, no es necesario que se esté en la pantalla del reproductor, en este caso se está reproduciendo desde una lista de canciones.
&
Pantalla del reproductor de música desde la que de puede descargar una canción, añadirla a favoritos ( opción de “me gusta”) . Desde esta pantalla también se podría compartirla en redes sociales. 
Esta pantalla se abre cuando se pulsa sobre una canción. \\
\hline
\end{tabular}

\begin{tabular}{ p{6cm} p{6cm}}
\hline
\\
\includegraphics[width=6cm]{Figures/android/PlayList.png}
&
\includegraphics[width=6cm]{Figures/android/artista.png} \\
\hline
\\
Pantalla que aparece al pulsar sobre una playlist, desde la que se pueden ver las canciones que hay en dicha lista de reproducción, el creador de esta, así como suscribirse a ella.
&
Pantalla de perfil de usuario (perfil de artista), en la cual aparecen las canciones de un usuario clasificadas por álbumes, así como las listas de reproducción creadas por este usuario. También aparecen enlaces con redes sociales y la posibilidad de suscribirte a este usuario.
En esta pantalla también aparecerían estadísticas del usuario como pueden ser numero de seguidores o el número de visitas o “me gusta” que poseen sus canciones.  
\\
\hline
\end{tabular}
% ______________________
% Aqui termina la tabla

\vspace{1cm}
Las pantallas relacionadas con la subida de canciones y registro de usuario no se han incluido debido a que serían formularios.
La pantalla mi perfil no se ha incluido, aunque esta estaría compuesta por la información personal así como estadísticas de la cuenta (número de reproducciones totales recibidos, número de me gusta totales recibidos). También se daría la opción de editar la información personal.

\subsection{Prototipo pantallas Web}

\begin{tabular}{ p{6cm} p{6cm}}
	\hline
	\\
	\includegraphics[width=6cm]{Figures/web/Login-web.png}
	&
	\includegraphics[width=6cm]{Figures/web/Lista-web.png} \\
	\hline
	\\
	Esta pantalla corresponde al inicio de sesión en la web y si es nuevo usuario, al registro del mismo.
	Se puede acceder mediante cuenta propia de la aplicación web (email o usuario) o mediante una cuenta de Google Plus.
	&
	Esta pantalla correspondería con la ventana principal en la aplicación, a traves de la cual se puede revisar la mejor música del momento recomendada para el usuario o acceder al resto de opciones (colección de álbunes, listas o pistas personales, búsquedas personalizadas o el propio perfil del usuario y su configuración).
	La aplicación permite al usuario reproducir la pista que desee haciendo click en sobre esta misma, procediendo así la aplicacíon a abrir en la parte inferior de la ventana la información de la pista y su estado actual (segundo de reproducción, si el usuario la ha marcado como favorita, etcétera).
	\\
	\hline
\end{tabular}

\begin{tabular}{ p{6cm} p{6cm}}
	\hline
	\\
	\includegraphics[width=6cm]{Figures/web/Main-web.png}
	&
	\includegraphics[width=6cm]{Figures/web/Reproduction-web.png} \\
	\hline
	\\
	Esta pantalla correspondería con tu colección de música en la aplicación, a traves de la cual se puede revisar la colección de álbunes, listas o pistas personales, así como las canciones que le gustan al usuario y sus canciones favoritas, ordenarlas o filtrarlas de manera personalizada.
	En este caso el usuario vería sus listas personales.
	La aplicación permite al usuario reproducir la pista que desee haciendo click en sobre esta misma, procediendo así la aplicacíon a abrir en la parte inferior de la ventana la información de la pista y su estado actual (segundo de reproducción, si el usuario la ha marcado como favorita, etcétera).
	&
	Esta pantalla correspondería con tu colección de música en la aplicación, donde a diferencia  de la anterior el usuario ha seleccionado visualizar las canciones que le gustan.
	La aplicación permite al usuario reproducir la pista que desee haciendo click en sobre esta misma, procediendo así la aplicacíon a abrir en la parte inferior de la ventana la información de la pista y su estado actual (segundo de reproducción, si el usuario la ha marcado como favorita, etcétera).
	\\
	\hline
\end{tabular}

\begin{tabular}{ p{6cm} p{6cm}}
	\hline
	\\
	\includegraphics[width=6cm]{Figures/web/Artist-web.png}
	&
	\includegraphics[width=6cm]{Figures/web/search-web.png} \\
	\hline
	\\
	Esta pantalla se correspondería con la vista general de un usuario donde este puede editar su foto de perfil si es el dueño de la cuenta o si es otro usuario suscribirse a las listas/pistas del usuario deseado.
	Se puede realizar scroll tanto en las canciones como listas o albunes del usuario.
	La aplicación permite al usuario reproducir la pista que desee haciendo click en sobre esta misma, procediendo así la aplicacíon a abrir en la parte inferior de la ventana la información de la pista y su estado actual (segundo de reproducción, si el usuario la ha marcado como favorita, etcétera).
	&
	Esta pantalla se correspondería con la obtención de resultados tras la búsqueda personalizada de un usuario que desea que le muestren toda la información disponible a cerca de las palabras clave seleccionadas, también se podría filtrar por pistas, listas, albunes o artistas disponibles.
	La aplicación permite al usuario reproducir la pista que desee haciendo click en sobre esta misma, procediendo así la aplicacíon a abrir en la parte inferior de la ventana la información de la pista y su estado actual (segundo de reproducción, si el usuario la ha marcado como favorita, etcétera).
	\\
	\hline
\end{tabular}
